\documentclass[12pt,a4paper]{report}
\usepackage[utf8]{inputenc}
\usepackage[margin=2cm]{geometry}
\usepackage{titling}

% Images
\usepackage{graphics}
\graphicspath{ {img/} }

% Automatically match quotation marks
\usepackage{csquotes}
\MakeOuterQuote{"}

% Bibleography
\usepackage[backend=biber]{biblatex}
\addbibresource{Dissertation.bib}

% Indent first paragraphs in sections and chapters
\usepackage{indentfirst}

% New commands
\newcommand\college{Clare College}


\title{Predicting events in computer games using machine learning}
\author{Artem Vasenin}
\date{\today}



\begin{document}
\maketitle

\section*{Proforma}
\begin{description}
\item[Name] \theauthor
\item[College] \college
\item[Title] Predicting arbitrary events in competitive computer team games
\item[Examination] %TODO
\item[Year] 2017
\item[Word Count] %TODO
\item[Project Originator] Artem Vasenin (av429)
\item[Project Supervisor] Yingzhen Li (yl494)
\end{description}

\subsection*{Original Project Aim}
The aim of this project was to develop an algorithm which could predict whether certain events would happen in multiplayer computer games.
In most modern multiplayer games, players are provided with a statistic after each match describing their performance in that match and summarising key choices they have made.
The project should be able to predict the values of such statistics.
The mean squared error of such predictions should be less than half of variance of the variable.
\subsection*{Summary of Work Completed}
A graphical model was created which produces a probability distribution for continuous statistics.
%TODO 
\subsection*{Special Difficulties}
%TODO

\section*{Declaration of Originality}
I, \theauthor\ of \college, being a candidate for Part II of the Computer Science Tripos, hereby declare that this dissertation and the work described in it are my own work, unaided except as may be specified below, and that the dissertation does not contain material that has already been used to any substantial extent for a comparable purpose. % This is one long-ass sentense
\\[1\baselineskip]
\noindent Signed 
\\[1\baselineskip]
\noindent Date: \thedate

\clearpage
\tableofcontents

\chapter{Introduction}
\section{Motivation}
Multiplayer computer games are becoming very popular, game's success often rests on how enjoyable it is.
Current method of improving player experience is to make sure that all players have equal chance of winning in a game.
For that purpose their skill has to be tracked and teams have to be arranged such that they are of equal strength.

In many popular games several roles have to be filled on each team for optimum performance.
Current algorithms, such as TrueSkill \cite{trueskill}, only track player's overall skill and do not consider what roles the player prefers and how good they are at each one.
This often leads teams being composed of players all wanting to play in the same role, which either leads to team under-performing or some players not enjoying the game as much as they could.
Moreover, I believe that the events that happen in game are more important to many player's experience than the actual outcome.

In addition, I believe that the system can improve overall prediction accuracy since tracking only overall player skill means that games with non-transitive strategies cannot be predicted properly. %TODO Make this clearer
For example, imagine three naive players playing the game of \emph{rock-paper-scissors}.
Player $R$ always plays \textit{rock}, player $P$ always plays \textit{paper} and player $S$ always plays \textit{scissors}.
When two of them play against each other, it is clear to us who will win, but a rating that only tracks overall skill can not predict the outcome properly.

To solve the above problems I wanted to make a system which would be able to predict different events that happen in game by tracking player skill in multiple areas.

\section{Related Work}
Before starting the project, I have looked into papers and articles about predicting events in sports and computer games.
I have found out that nothing similar has been tried before.
Most closely related algorithms (such as \cite{trueskill} and \cite{bayesianranking}) were made to only predict the outcome of the game, not any related statistic.
Outside of academia, other algorithms were made which used machine learning to improve their performance, but again only focusing on the outcome of the game.
%TODO Write more here
\chapter{Preparation}
\section{Background Research}
Since I decided to use machine learning in my project I had to get some background in the area.
Unfortunately, computer science course is in Lent term, therefore I attended a similar course in Engineering department in Michaelmas term, before beginning work on the theory.
\subsection*{Programming Language and ML Libraries}
Machine learning is one of the key elements of this project, therefore I needed a library that would implement key techniques used in ML for me, so that I don't have to write them myself.
I have compared a number of libraries (such as Tensorflow\footnote{tensorflow.org/} and Torch\footnote{torch.ch/}), I was primarily interested in how much funcitonality they provide, their performance and how easy it is to learn them.

I have compared their functionality by completing standard machine learning task, such as creating a neaural network to classify images in the MNIST\footnote{yann.lecun.com/exdb/mnist/} dataset.
Performance was compared by timing how long it would take to achieve 99\% accuracy, in most cases it came down whether GPU acceleration was supported.
Since I would have to learn how to use the chosen library in detail I also paid attention to the amount of support material available.
I took note of quality of documentation and also compared number of questions and answers on StackOverflow.

In the end I have arrived at the conclusion that they all provided required functionality and were 
sufficiently easy to use.
API for most of them were written in different languages.
I did not want to learn a new language in addition to learning a library, therefore I decided to use TensorFlow, API for which was written in Python a language I am most comfortable writting code in.
To a lesser extent Python was also chosen since it has a package manager\footnote{pypi.python.org/pypi/pip/}, which makes it much easier to install additional packages.
\section{Requirement Analysis}

\chapter{Implementation}
\chapter{Evaluation}
\chapter{Conclusions}

\printbibliography
\appendix



\end{document}